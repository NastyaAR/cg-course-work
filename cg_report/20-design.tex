\chapter{Конструкторская часть}

В данном разделе представлены требования к программному обеспечению, даны описания алгоритмов, выбранных для решения поставленной задачи, приведены выбранные структуры данных.

\section{Требования к программному \newlineобеспечению}

Программное обеспечение должно предоставлять следующие возможности для пользователя:
\begin{itemize}
	\item[---] изменение скорости движения шарика;
	\item[---] изменение цвета шарика, качелей и фона;
	\item[---] добавление направленных источников света;
	\item[---] изменение оптических свойств шарика и качелей.
\end{itemize}

Выделяются следующие требования к программному обеспечению:
\begin{itemize}
	\item[---] входные данные для построения объектов задаются в файлах;
	\item[---] кадры должны генерироваться со скоростью 30 кадров в секунду \footnote{Для приложений визуализации комфортный уровень fps --- до 40 кадров в секунду \cite{fps}}.
\end{itemize}

\section{Функциональная модель программы \newline первого уровня в нотации idef0}

Функциональная модель программы в нотации IDEF0 представлена на рисунке \ref{img:idef1}:

\img{11cm}{idef1}{Функциональная модель первого уровня}
\pagebreak

\section{Структуры данных}

В таблице \ref{tbl:table1} представлены описания структур данных для объектов:

\tableimg{9.5cm}{table1}{Структуры данных объектов}

\section{Схема алгоритма, использующего \newline Z-буфер}

На рисунке \ref{img:zbuff} приведена схема алгоритма, использующего Z-буфер.

\img{190mm}{zbuff}{Схема алгоритма, использующего Z-буфер}

\pagebreak

\section{Схема алгоритма построения карты \newlineтеней}

На рисунке \ref{img:shadows} приведена схема алгоритма построения карты теней.

\img{190mm}{shadows}{Схема алгоритма построения карты теней}

\pagebreak

\section{Схема алгоритма закраски Гуро с \newlineмоделью освещения Фонга}

На рисунке \ref{img:intensity} приведена схема алгоритма построения карты теней.

\img{120mm}{intensity}{Схема алгоритма закраски Гуро с моделью освещения Фонга}

\pagebreak

\section*{Вывод}

В данном разделе представлены схема алгоритма, использующего Z-буфер, алгоритма построения карты теней, алгоритма закраски Гуро с моделью освещения Фонга, также представлены требования к разрабатываемому программному обеспечению и обеспечиваемые им возможности, описаны структуры данных и приведена модель программы первого уровня в нотации idef0.
