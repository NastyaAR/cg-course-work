\chapter*{ВВЕДЕНИЕ}
\addcontentsline{toc}{chapter}{ВВЕДЕНИЕ}

\fontsize{16}{19}\selectfont

Одна из задач компьютерной графики --- визуализация движения объектов на экране.

\textbf{Целью} курсовой работы является разработка программного обеспечения для визуализации движения шарика на качелях.

Для достижения поставленной цели необходимо решить следующие \textbf{задачи}:

\begin{enumerate}
	\item[---] формализовать задачу в виде IDEF0-диаграммы;
	\item[---] описать и проанализировать алгоритмы построения теней, удаления невидимых рёбер и граней, алгоритмы закрашивания;
	\item[---] спроектировать программное обеспечение;
	\item[---] выбрать средства реализации и реализовать спроектированное программное обеспечение; 
	\item[---] исследовать зависимость времени генерации кадра от количества источников света.
\end{enumerate}

\pagebreak
